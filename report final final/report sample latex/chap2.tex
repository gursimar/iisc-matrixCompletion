\chapter{Formulas and theorems}\label{sec:formulas}
Each main chapter or section should start with a short description of
what it holds, and why. Top tip --- begin the whole writing enterprise
with a first draft of this little bit for each chapter. It will force
you to think about overall structure.
\par
Here we get going with mathematics and show firstly some simple
equations and then some that are slightly less simple. Then there's a
theorem whose proof is relegated to an appendix, followed by some
numbered results.
%
\section{Plain}
Here's an inline formula: \(E=mc^2\), and here's the same
thing displayed:\[E=mc^2.\]A matrix needs to be displayed:
\[  \det\left(\begin{array}{cc}
        a & b \\  c & d \\
      \end{array}\right) = ad-bc,   \]
and here's a result that's displayed and labelled:
\begin{equation}\label{eq:display}
\lim_{a\to\infty}\int_0^a\exp(-x^2)\,dx=\fr12\sqrt\pi.
\end{equation}Note the punctuation in eq.~(\ref{eq:display}) \etc,
which recognises that displayed equations are parts of sentences.
\par
Note also that a smaller (\verb+\textstyle+) size of numerical fraction
is often useful in a displayed equation. The appropriate
\verb+\newcommand+ called \verb+\fr+ is defined in the
\Quote{preamble}, in file \texttt{report.tex}.
\par
More complicated limits on \verb+\int+ (and \verb+\sum+) need to
be enclosed in braces $\{\cdots\}$.
\par
In an inline formula, avoid ugly fractions such as \(v=\frac st\) and
\(\frac{dy}{dx}=x+y\). It looks much better to write \(v=s/t\) and
\(dy/dx=x+y\). That is, keep \verb+\frac+ for
display:\[\frac{dy}{dx}=x+y\]where it belongs.
\par
Beware of inadvertent blank lines in your \texttt{.tex} file; they often
creep in immediately after a displayed equation. A blank line gives a
new paragraph, usually indented, which may not be what you want. You can 
avoid blank lines altogether by breaking paragraphs with \verb+\par+
instead, as done here.
%
\section{Fancy}
Now an equation on several lines using \verb+align+:
\begin{align*}
  |\vec A|^2 &= a_1^2+a_2^2  \\
             &= \sin^2\theta+\cos^2\theta \\
             &= 1
\end{align*}--- where the star on \verb+align*+ suppresses
all line-numbering\footnote{And re-phrase material where any displayed
equation splits between pages.}.
\par
Again, with just one line numbered for reference:
\begin{equation}
\begin{aligned}
  |\vec A|^2 &= a_1^2+a_2^2  \label{line-one}\\
             &= \sin^2\theta+\cos^2\theta \\
             &= 1 
\end{aligned}
\end{equation}
Now you can refer to equation~(\ref{line-one}).
\par
If you want to number the equations in Chapter 2 as (2.1), (2.2), etc,
then\footnote{Top tip: easily find out about any \LaTeX\ command by
putting it \textit{with} its backslash (and perhaps the word \lq latex')
into your favourite search engine.} put \lq\verb+\numberwithin+' into
\textsl{Google} or see \cite[Sec 8.2.14]{MG}.
\par
Notice that in equations you use \verb+\lim+, \verb+\exp+, \verb+\sin+
and \verb+\cos+ and not plain $lim$, $exp$, $sin$ and $cos$. See
\cite[Sec.~3.3]{NSS} for a list.
\par
For a lot of complicated multi-line formulas it's better to use the more
sophisticated display environments provided by the package
\texttt{amsmath} --- see Sec.~8.2 of \comp\ \cite{MG}.
For instance the \texttt{cases} environment is used to get
\[\theta(x)=\begin{cases}0&\text{if $x<0$,}\\
		1&\text{if $x>0$.}\end{cases}\]
And if you need continued fractions, instead of 
\[x=\frac{1}{a_2+\frac{1}{a_3+\frac{1}{a_4+\cdots}}},\]a better-looking
result
is\[x=\frac{\strut1}{\displaystyle a_2+\frac{\strut1}{\displaystyle
a_3+\frac{\strut1}{\displaystyle a_4+\cdots}}},\]for which
\texttt{amsmath}
provides the command \verb+\cfrac+ \cite[Sec.~8.4.2]{MG}.
\par
The package \texttt{amssymb} \cite[Chap.~8]{MG} extends the range of
mathematical symbols (\eg\(\mathbb{R}\), \(\mathbb{Z}\) and
\(\mathbb{N}\) are available).
\par
And its sister \texttt{amstext} provides the command \verb+\text+ to
put words into a displayed equation \[ \text{like this: }E=mc^2. \]Note
that you need to put in by hand the spacing between the text and the
mathematical symbols.
%
\section{Theorems}\label{angels}
This is theorem~\ref{angels}.
\begin{description}
\item[Theorem]: The whole is greater than the sum of its parts.
\item[Proof:] See App.~\ref{app:proofs}, Sec.~\ref{pf:angels}.
\end{description}
For numbered theorems, use \verb+\newtheorem+ \cite[Sec.~3.8]{NSS}.
\par
Here's the main sequence of theorems.
\newtheorem{Main}{Theorem} % this could go into the preamble
\begin{Main}[my first result]
The first theorem, showing
\[ \bar D^n\quad\text{and}\quad\bar{D^n} \]
where the first bar is over just the \lq D\rq\ and the second is over
the whole expression. Also the word \lq and\rq\ is inside displayed
maths, using \verb+\text+ with extra space before and after.
\end{Main}And that's how to do left and right quotation marks, too,
with \verb+\lq+ and \verb+\rq+.
\begin{Main}
The second theorem --- with \verb+\overline+ inline: $\overline{D^n}$.
\end{Main}
\begin{Main} the third theorem\end{Main}
Here's another sequence of results.
\newtheorem{second}{Lemma} % this too could go into the preamble
\begin{second}[my second result]
The next result (a mere lemma) \dots\end{second}
\begin{second}\dots\ and the next\end{second}
\comp\ \cite[Sec.~3.3.3]{MG} describes how to use the
packages \texttt{amsmath} and \texttt{amsthm} to state (and
cross-reference) theorems, lemmas, definitions, proofs, \etc\ in a more 
sophisticated way.
%
\section{Summary}
Each chapter should end with a round-up of its contents and a link
with the contents of the next.
\par
After formulas, equations and theorems, the next important
topics are graphics and tables.
